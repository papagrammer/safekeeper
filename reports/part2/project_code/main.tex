\documentclass{article}
\usepackage[LGR,T1]{fontenc}
\usepackage[utf8]{inputenc}
\usepackage[greek, english]{babel}
\usepackage{alphabeta}
\usepackage{natbib}
\usepackage{graphicx}
\usepackage{biblatex}
\addbibresource{references.bib}

\def\code#1{\texttt{#1}}

\usepackage{eso-pic}% http://ctan.org/pkg/eso-pic
\usepackage{lipsum}% http://ctan.org/pkg/lipsum

\title{Project Code-v0.1}

\author{\\
\includegraphics[width=3in]{safeguard}\\[1ex]\\\\
}

\begin{document}

\maketitle

\newpage


Θεόδωρος Ντάκουρης - ntakouris@ceid.upatras.gr - ΑΜ 1054332 : Editor
\\

\begin{tabular}{|l|c|c|}
\hline
Όνοματεπώνυμο & email & Αριθμός μητρώου  \\
\hline
Θεόδωρος Ντάκουρης & ntakouris@ceid.upatras.gr & 1054332 \\
Βασίλειος Βασιλόπουλος & vvasil@ceid.upatras.gr &  1054410 \\
Νικόλαος Σουλτάνης & soultanis@ceid.upatras.gr & 1054319  \\
Βάιος Λασκαρέλιας & laskarelias@ceid.upatras.gr & 1054432 \\
Αντόν Παπά & papa@ceid.upatras.gr & 1054337 \\
\hline
\end{tabular}

\renewcommand{\contentsname}{Περιεχόμενα}
\tableofcontents

\section{Github}

Το project της ομάδας είναι στην τοποθεσία:

\url{https://github.com/ceidhub/safekeeper}
\\
Ο κώδικας βρίσκεται στο directory src.

\section{Εργαλεία}
Χρησιμοποιήθηκαν:
\begin{itemize}
    \item \LaTeX/Overleaf.com - Συγγραφή του παρόντος τεχνικού κειμένου
    \item Photoshop - Φωτογραφία Σελίδας Τίτλου
\end{itemize}


\end{document}
