\documentclass{article}
\usepackage[LGR,T1]{fontenc}
\usepackage[utf8]{inputenc}
\usepackage[greek, english]{babel}
\usepackage{alphabeta}
\usepackage{natbib}
\usepackage{graphicx}
\usepackage{biblatex}
\addbibresource{references.bib}

\def\code#1{\texttt{#1}}

\usepackage{eso-pic}% http://ctan.org/pkg/eso-pic
\usepackage{lipsum}% http://ctan.org/pkg/lipsum

\title{Use Cases - v0.1}

\author{\\
\includegraphics[width=3in]{safeguard}\\[1ex]\\\\
}

\begin{document}

\maketitle

\newpage

\noindentΑντόν Πάπα - ΑΜ 1054337 : Editor \\
Θεόδωρος Ντάκουρης - AM 1054332: Editor, Reviewer\\\\
Όλα τα μέλη της ομάδας συνεισέφεραν σε αυτό το έγγραφο.
\\

\begin{tabular}{|l|c|c|}
\hline
Όνοματεπώνυμο & email & Αριθμός μητρώου  \\
\hline
Θεόδωρος Ντάκουρης & ntakouris@ceid.upatras.gr & 1054332 \\
Βασίλειος Βασιλόπουλος & vvasil@ceid.upatras.gr &  1054410 \\
Νικόλαος Σουλτάνης & soultanis@ceid.upatras.gr & 1054319  \\
Βάιος Λασκαρέλιας & laskarelias@ceid.upatras.gr & 1054432 \\
Αντόν Πάπα & papa@ceid.upatras.gr & 1054337 \\
\hline
\end{tabular}

\renewcommand{\contentsname}{Περιεχόμενα}
\tableofcontents

\newpage

\section{Use Cases}

\subsection{RFID Check In-Out}

\noindent Actor: 
\begin{itemize}
    \item RFID card reader
\end{itemize}

\noindent Βασική ροή:
\begin{itemize}
    \item Κατά την επαφή-διάβασμα της κάρτας
    \item Έλεγχοι: 
    \begin{itemize}
        \item Τύπος εισόδου (security - guest - employee)
        \item Έχει βάρδια ο εργαζόμενος
        \item Έχει λήξει η κάρτα 
        \item Σωστό πόστο-περιοχή εισόδου
    \end{itemize}
    \item Χαρακτηριστικός υψίτονος ήχος
\end{itemize}

\noindent Εναλλακτική ροή: 
\begin{itemize}
    \item Προβολή μηνύματος λάθους μαζί με το λόγο απόρριψης στο notifications tab
\end{itemize}

\noindent Περαιτέρω εξήγηση:
\begin{itemize}
\item RFID Check In = RFID Basic Access Check
\item RFID Check Out = RFID Basic Access Check + Εάν έχει κάνει σωστά check in ο εργαζόμενος
\end{itemize}

\newpage
\noindent RFID Check In-Out: \\
\noindent\makebox[\textwidth]{\includegraphics[width=\paperwidth]{rfid-access-checks.png}}
\newpage

\subsection{Insert-Delete Employee}

\noindent Actor:
\begin{itemize}
    \item Security guy
\end{itemize}

\noindent Βασική ροή:
\begin{itemize}
    \item Aναζήτηση με:
    \begin{itemize}
        \item Employee ID
        \item Κάρτα RFID
        \item Ονοματεπώνυμο
    \end{itemize}
    \item Εισαγωγή λόγου διαγραφής - εγγραφής
    \item Εισαγωγή Εγγράφου από γραμματεία - HR
    \item Απενεργοποίηση - διαγραφή κάρτας με επιβεβαίωση
    \item PIN Entry Check
\end{itemize}

\noindent Εναλλακτική ροή 1: 
\begin{itemize}
    \item Εισαγωγή εργαζομένου με ID που υπάρχει ήδη
    \item Εμφανίζεται μήνυμα λάθους και επαναλαμβάνεται η διαδικασία
\end{itemize}

\noindent Εναλακτική ροή 2:
\begin{itemize}
    \item Η κάρτα RFID ανήκει σε άλλον / δεν έχει διαγραφεί σωστά / είναι corrupted / έχει χαλάσει ολοκληρωτικά
    \item Επαναλαμβάνεται η διαδικασία με νέα κάρτα ή αφού σβηστούν τα στοιχεία της κάρτας, εάν έχει ξεχαστεί
\end{itemize}

\noindent Εναλλακτική ροή 3:
\begin{itemize}
    \item Διαγραφή κάποιου εργαζομένου που υπάρχει ήδη
\end{itemize}

\noindent Εναλλακτική ροή 4:
\begin{itemize}
    \item Επιλέγει να εισάγει guest, οπότε δεν εισάγεται employee, απλά εκδίδεται μια προσωρινή κάρτα. 
\end{itemize}

\newpage
\noindent Διάγραμμα Insert Delete Employee: \\
\noindent\makebox[\textwidth]{\includegraphics[width=\paperwidth]{use_cases_insert_delete_employee.png}}
\newpage

\subsection{Access Log}

\noindent Actor:
\begin{itemize}
    \item Chief security guy
\end{itemize}

\noindent Βασική ροή:
\begin{itemize}
    \item Προβολή τελευταίων incident
    \item Αναζήτηση με βάση:
    \begin{itemize}
        \item incident type (alarm / check in-out / security incident)
        \item incident ID
        \item χρονικές περιόδους
        \item RFID κάρτα εργαζομένου
    \end{itemize}
    \item Τα στοιχεία προβάλλονται στο UI
\end{itemize}

\noindent Εναλλακτική ροή 1:
\begin{itemize}
    \item Δεν εμφανίζεται κανένα incident
\end{itemize}

\noindent Εναλλακτική ροή 2:
\begin{itemize}
    \item Το access log έχει καταστραφεί
    \item Εμφανίζεται μήνυμα για επικοινωνία με τεχνική υποστήριξη
\end{itemize}

\newpage
\noindent Διάγραμμα Access logs and incidents: \\
\noindent\makebox[\textwidth]{\includegraphics[width=\paperwidth]{use_cases_access_log.png}}
\newpage

\subsection{Drone notifications}

\noindent Actor:
\begin{itemize}
    \item Drone
\end{itemize}

\noindent Βασική ροή:
\begin{itemize}
    \item Αναγνώριση συμβάντος
    \item Αναγνώριση πιθανών παραγόντων συναγερμού:
    \begin{itemize}
        \item άτομα
        \item ζώα 
        \item φωτιά
        \item μηχανοκίνητα οχήματα
    \end{itemize}
    \item Έναρξη video livestream 
    \item Ταυτόχρονη λήψη φωτογραφιών, σε τακτά χρονικά διαστήματα
    \item Ταυτόχρονη αποστολή φωτογραφιών
\end{itemize}

\noindent Εναλλακτική ροή 1:
\begin{itemize}
    \item Αποτυχία σύνδεσης με κέντρο ελέγχου
    \item Αυτόματο recall
    \item Ειδοποίηση μόλις αποκατασταθεί η σύνδεση
\end{itemize}

\noindent Εναλλακτική ροή 2:
\begin{itemize}
    \item Inactivity για παραπάνω από 2 λεπτά στην τοποθεσία συμβάντος
    \item Αυτόματο recall
\end{itemize}

\newpage
\noindent Διάγραμμα drone notifications: \\
\noindent\makebox[\textwidth]{\includegraphics[width=\paperwidth]{use_cases_drone_notifications.png}}
\newpage

\subsection{Incident Submission}

\noindent Actor:
\begin{itemize}
    \item Notification Subsystem
    \item Security guy
\end{itemize}

\noindent Βασική ροή:
\begin{itemize}
    \item Μετά την αναγνώριση περιστατικού από τις φωτογραφίες του drone:
    \begin{itemize}
        \item Αυτόματη εισαγωγή της τρέχουσας ώρας
        \item Εισαγωγή κύριου συμβάντος
        \item Εισαγωγή επιπλέων σχολίων
    \end{itemize}
    \item Επικύρωση με αριθμό PIN του επιβλέπων εκείνη τη στιγμή του συμβάντος
    \item Αποστολή τελικής έκδοσης του συμβάντος
\end{itemize}

\noindent Εναλλακτική ροή 1:
\begin{itemize}
    \item Ο αριθμός PIN αποτυγχάνει (3 φορές λάθος όπως αναφέρεται στο “Pin Entry”)
    \item Αποστέλλεται ανώνυμο report με όλα τα στοιχεία που υπάρχουν διαθέσιμα στα κεντρικά, ώστε να εξεταστεί “χειροκίνητα” το συμβάν
\end{itemize}

\newpage
\noindent Διάγραμμα Incident submission: \\
\noindent\makebox[\textwidth]{\includegraphics[width=\paperwidth]{use_cases_incident_submission.png}}
\newpage

\subsection{Send drone}

\noindent Actor:
\begin{itemize}
    \item Perimeter Monitoring System
    \item Security guy
\end{itemize}

\noindent Βασική ροή:
\begin{itemize}
    \item Ειδοποίηση από κάμερα ή αισθητήρα για συγκεκριμένο συμβάν
    \item Αναζήτηση τοποθεσίας συμβάντος
    \item Αποστολή Drone και λήψη πληροφοριών (έναρξη use case “drone notifications”)
    \item Αποστολή ειδοποίησης στα κεντρικά
    \item Recall drone on demand
    \item Use case “Incident Submission”
\end{itemize}

\noindent Εναλλακτική ροή 1:
\begin{itemize}
    \item Επιλογή να μην σταλεί drone
\end{itemize}

\noindent Εναλλακτική ροή 2:
\begin{itemize}
    \item Αποτυχία εκκίνησης drone ή πρόβλημα κατά τη διάρκεια πτήσης
    \item Αυτόματο recall εάν γίνεται
    \item Αποστολή τοποθεσίας drone στα κεντρικά (αν δεν γίνει recall)
    \item Use case “Incident Submission”
\end{itemize}

\newpage
\noindent Διάγραμμα send drone: \\
\noindent\makebox[\textwidth]{\includegraphics[width=\paperwidth]{use_cases_send_drone.png}}
\newpage

\subsection{Silent alarm}

\noindent Actor:
\begin{itemize}
    \item Security guy
\end{itemize}

\noindent Βασική ροή:
\begin{itemize}
    \item Εισαγωγή περιγραφής (προαιρετικά)
    \item Εισαγωγή PIN εργαζομένου
    \item Αποστολή στα κεντρικά
\end{itemize}


\noindent Διάγραμμα Silent alarm: \\
\noindent\makebox[\textwidth]{\includegraphics[width=\paperwidth]{use_cases_silent_alarm.png}}
\newpage

\subsection{Pin Entry}

\noindent Actor:
\begin{itemize}
    \item Security guy
\end{itemize}

\noindent Βασική ροή
\begin{itemize}
    \item Prompt για PIN entry 
    \item Έλεγχοι:
    \begin{itemize}
        \item Σωστό PIN εργαζομένου
        \item Έχει κάνει check in προηγουμένως
        \item Είναι το PIN entry ενεργοποιημένο από τα κεντρικά
    \end{itemize}
    \item Διπλή Επιβεβαίωση Ενέργειας πριν το τελικό action
    \item Επιστροφή σε οποιαδήποτε ενέργεια ζήτησε PIN access επικύρωση
\end{itemize}

\noindent Εναλλακτική ροή 1:
\begin{itemize}
    \item Ο χρήστης ακυρώνει τη διαδικασία
\end{itemize}

\noindent Εναλλακτική ροή 2:
\begin{itemize}
    \item Ο χρήστης βάζει τον κωδικό λανθασμένα
    \item Ξαναπροσπαθεί 3 φορές
    \item Αν αποτύχει σε όλες, στέλνεται ειδοποίηση στα κεντρικά για επανενεργοποίηση της δυνατότητας χρήσης του PIN
\end{itemize}

\newpage
\noindent Διάγραμμα Pin entry: \\
\noindent\makebox[\textwidth]{\includegraphics[width=\paperwidth]{use_cases_pin.png}}
\newpage

\section{Εργαλεία}
Χρησιμοποιήθηκαν:
\begin{itemize}
    \item \LaTeX/Overleaf.com - Συγγραφή του παρόντος τεχνικού κειμένου
    \item Photoshop - Φωτογραφία Σελίδας Τίτλου
    \item draw.io - Σχεδιασμός διαγραμμάτων
\end{itemize}


\end{document}
