\documentclass{article}
\usepackage[LGR,T1]{fontenc}
\usepackage[utf8]{inputenc}
\usepackage[greek, english]{babel}
\usepackage{alphabeta}
\usepackage{natbib}
\usepackage{graphicx}
\usepackage{biblatex}
\addbibresource{references.bib}

\def\code#1{\texttt{#1}}

\usepackage{eso-pic}% http://ctan.org/pkg/eso-pic
\usepackage{lipsum}% http://ctan.org/pkg/lipsum

\title{Project Code-v0.2}

\author{\\
\includegraphics[width=3in]{safeguard}\\[1ex]\\\\
}

\begin{document}

\maketitle

\newpage


Θεόδωρος Ντάκουρης - ntakouris@ceid.upatras.gr - ΑΜ 1054332 : Editor
\\

\begin{tabular}{|l|c|c|}
\hline
Όνοματεπώνυμο & email & Αριθμός μητρώου  \\
\hline
Θεόδωρος Ντάκουρης & ntakouris@ceid.upatras.gr & 1054332 \\
Βασίλειος Βασιλόπουλος & vvasil@ceid.upatras.gr &  1054410 \\
Νικόλαος Σουλτάνης & soultanis@ceid.upatras.gr & 1054319  \\
Βάιος Λασκαρέλιας & laskarelias@ceid.upatras.gr & 1054432 \\
Αντόν Παπά & papa@ceid.upatras.gr & 1054337 \\
\hline
\end{tabular}

\renewcommand{\contentsname}{Περιεχόμενα}
\tableofcontents

\section{Αλλαγές}
\subsection{v0.2}
\begin{itemize}
    \item Προστέθηκαν δηλωμένες κλάσεις domain model μαζί με μερικά attributes.
    \item Προστέθηκαν παρακάτω οι σημειώσεις περί των περιεχομένων του github repository
\end{itemize}

\section{Github Link}

Το project της ομάδας είναι στην τοποθεσία:

\url{https://github.com/ceidhub/safekeeper}
\\
Ο κώδικας βρίσκεται στο directory src.

\section{Σημειώσεις περί περιεχομένων}
Στο directory src βρίσκεται ο κώδικας, τα αρχεία πρωτοτύπων του UI απο pyqt5, τα μεταφρασμένα αρχεία σε python για το UI και το business logic στις αντίστοιχες κλάσεις του domain.


Στο directory meeting βρίσκονται σημειώσεις οργάνωσης της ομάδας που συμπληρώνονται στην πρώτη συνάντηση πριν αρχίσει η δουλειά σε κάθε παραδοτέο.


Στο directory reports βρίσκονται τα \LaTeX αρχεία μαζί με resources και τα αντίστοιχα exported pdfs.


Κάθε μέλος κάνει pull requests για να βάλει νέο περιερχόμενο, αναφέροντας το αντίστοιχο ανοιγμένο issue προς κλείσιμο.


Milestone \% tracking γίνεται από την καρτέλα των Projects.


Σε κάθε παραδοτέο ανεβάζουμε ένα zip τα ζητούμενα ως 'release'.


Η ομάδα είναι οργανωμένη σε ένα organization ώστε να έχουν τα κατάλληλα access rights.

\section{Εργαλεία}
Χρησιμοποιήθηκαν:
\begin{itemize}
    \item \LaTeX/Overleaf.com - Συγγραφή του παρόντος τεχνικού κειμένου
    \item Photoshop - Φωτογραφία Σελίδας Τίτλου
\end{itemize}


\end{document}
