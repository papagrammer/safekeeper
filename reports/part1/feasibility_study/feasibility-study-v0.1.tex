\documentclass{article}
\usepackage[LGR,T1]{fontenc}
\usepackage[utf8]{inputenc}
\usepackage{lmodern,textcomp}
\usepackage[greek, english]{babel}
\usepackage{alphabeta}
\usepackage{natbib}
\usepackage{graphicx}
\usepackage{biblatex}
\usepackage{tabularx}
\usepackage{longtable}
\addbibresource{references.bib}
\renewcommand\tabularxcolumn[1]{m{#1}}% for vertical centering text in X column
\usepackage[section]{placeins}

\def\code#1{\texttt{#1}}

\usepackage{eso-pic}% http://ctan.org/pkg/eso-pic
\usepackage{lipsum}% http://ctan.org/pkg/lipsum

\usepackage[section]{placeins}

\title{Feasibility Study - v0.1}

\author{\\
\includegraphics[width=3in]{safeguard}\\[1ex]\\\\
}

\begin{document}

\maketitle
\newpage

\begin{center}
Λασκαρέλιας Βάιος - laskarelias@ceid.upatras.gr - ΑΜ 1054432 : Editor \\
Πάπα Αντόν - papa@ceid.upatras.gr - ΑΜ 1054337 : Editor
\end{center}

\begin{tabular}{|l|c|c|}
\hline
Όνοματεπώνυμο & email & Αριθμός μητρώου  \\
\hline
Θεόδωρος Ντάκουρης & ntakouris@ceid.upatras.gr & 1054332 \\
Βασίλειος Βασιλόπουλος & vvasil@ceid.upatras.gr &  1054410\\
Νικόλαος Σουλτάνης & soultanis@ceid.upatras.gr & 1054319  \\
Βάιος Λασκαρέλιας & laskarelias@ceid.upatras.gr & 1054432 \\
Αντόν Πάπα & papa@ceid.upatras.gr & 1054337 \\
\hline
\end{tabular}
\renewcommand{\contentsname}{Περιεχόμενα}

\tableofcontents

\newpage
\thispagestyle{plain} % empty
\mbox{}
\newpage

\section{Γενική εφικτότητα}
Το Safeguard πρόκειται για ένα surveillance και access control σύστημα, εμπλουτισμένο με drones “παρακολουθητές”. Προορίζεται για μικρά έως μεσαία κτίρια που απαιτούν ένα custom ή offline σύστημα ασφαλείας με μικρό κόστος σε σχέση με τον ανταγωνισμό. Στοχεύει να είναι εύκολο στην χρήση και ταυτόχρονα να προσφέρει ασφάλεια, χρησιμοποιώντας τις πιο πρόσφατες τεχνολογίες ανοιχτού λογισμικού. Παρόμοια εμπορικά συστήματα είναι πάντα τρέχοντα στην αγορά, παρόλο που δεν χρησιμοποιούν πρόσφατες τεχνολογίες. Έτσι, θεωρούμε πως το Safeguard αποτελεί ανταγωνιστικό προϊόν στον τομέα του.\\ \\
Η εφικτότητα της ανάπτυξης του Safeguard εξαρτάται από πολλούς παράγοντες. Οι παράγοντες μπορούν να αναλυθούν περαιτέρω σε 5 κατηγορίες: \begin{itemize}
\item Financial Feasibility
\item Technical Feasibility
\item Resource and Time Feasibility
\item Risk Feasibility
\item Legal Feasibility
\end{itemize}

Συνοπτικά, το έργο θεωρείται εφικτό λόγω του χαμηλού του αρχικού κόστους, την χρήση διαδεδομένων εργαλείων, την συνεργατικότητα και την ετοιμότητα της ομάδας, και φυσικά την νομική και ηθική του ορθότητα. \\
	Παρακάτω γίνεται ανάλυση της εφικτότητας του έργου, με βάση τις προαναφερθείσες κατηγορίες.
\section{Financial Feasibility (Οικονομική εφικτότητα)}
	Στο σύστημα θα χρησιμοποιηθούν drones ως “παρατηρητές” και card readers τα οποία ουσιαστικά θα είναι και τα μόνα που θα χρειαστούν όσον αφορά το υλικό. Θα πρέπει όμως να γίνεται διαρκής συντήρηση και αλλαγή εξαρτημάτων ή ολόκληρων συσκευών, εφόσον αυτό είναι απαραίτητο, συνεπώς θα υπάρχει επιπλέον κόστος για το maintainability του υλικού. Παρόλα αυτά, το κόστος για όλα τα παραπάνω είναι φυσιολογικό και σχεδόν αμελητέο συγκριτικά με τα αναμενόμενα κέρδη.
	Το software που θα χρησιμοποιηθεί για την υλοποιηση του Safeguard, θα είναι open-source. Επομένως το software, αλλά και το maintainability του, δεν θα έχει καμία επιβάρυνση στον προϋπολογισμό, καθώς τα open-source λογισμικά είναι δωρεάν. \\
	Από τα παραπάνω, αφού το αρχικό κόστος είναι περιορισμένο, συμπεραίνεται ότι η υλοποίηση του Safeguard είναι οικονομικά εφικτή (Financially Feasible).
	
\newpage 

\section{Technical Feasibility (Τεχνολογική εφικτότητα)}
Το τελικό προϊόν θα είναι ένα ολοκληρωμένο σύστημα βασισμένο σε stable, open source τεχνολογίες. Μερικές βασικές τεχνολογίες από αυτές είναι:
\begin{itemize}
\item Python
\item MySql
\item Qt
\item	Drone Control Library (e.g. Skydio)
\item	Card Reader Integration
\end{itemize}
Για την συγγραφή των τεχνικών κειμένων και για την διαχείριση αρχείων χρησιμοποιούνται δωρεάν εργαλεία όπως:
\begin{itemize}
\item Google Docs
\item Google Drive
\item Overleaf
\item Draw.io
\end{itemize}
Και για την διαχείριση κώδικα, επικοινωνία της ομάδας, και ανάθεση / διαχείριση έργου θα χρησιμοποιηθούν τα εργαλεία:
\begin{itemize}
\item GitHub
\item Discord
\end{itemize}
Η πλειοψηφία των τεχνολογιών που αξιοποιούνται για την υλοποίηση του συστήματος έχουν ξαναχρησιμοποιηθεί από τα μέλη της ομάδας, και υπάρχει ήδη η σχετική τεχνική κατάρτιση για τα εργαλεία που χρησιμοποιούνται. Τα εργαλεία λοιπόν δίνουν την δυνατότητα να παραχθεί έργο είτε ατομικά, είτε σε συνεργασία με τα άλλα μέλη κάνοντας το αποτελεσματικότερο και ευκολότερο. Με βάση προηγούμενη εμπειρία, θεωρούμε πως με την χρήση γνώριμων εργαλείων είναι εφικτό να ανταπεξέλθουμε στα χρονικά περιθώρια. \\ 
Συνεπώς, αφού στηριζόμαστε σε δωρεάν, δοκιμασμένες τεχνολογίες και σε γνώριμα, αποτελεσματικά εργαλεία, η υλοποίηση είναι τεχνολογικά εφικτή (Technically Feasible).
\newpage
\section{Resource and Time Feasibility (Εφικτότητα Πόρων και Χρόνου)}
Για την επιτυχή υλοποίηση της εφαρμογής χρειάζονται:
\begin{itemize}
\item Χώρος εργασίας εξοπλισμένος με υπολογιστές (απλό Desktop ή Laptop)
\item Το προαναφερθέν Hardware (Drone, Card Reader)
\item Τα προαναφερθέντα εργαλεία (Google Drive, Google Docs, Overleaf, Draw.io, GitHub,\item Discord), τα οποία είναι δωρεάν διαθέσιμα
\item Χώρος για testing των φυσικών τμημάτων του Safeguard.
\end{itemize}
	Εφόσον τα εργαλεία υπάρχουν και το απαραίτητο υλικό έχει περιορισμένο κόστος, το Safeguard διαθέτει τους απαραίτητους πόρους για να έρθει εις πέρας.
\section{Risk Feasibility (Εφικτότητα με βάση τα ρίσκα)}
Τα ρίσκα που αφορούν την υλοποίηση του Safeguard μπορούν να αναπτυχθούν πιο αναλυτικά σε:
\subsection{Ρίσκα Επεκτασιμότητας}
Όλο το λογισμικό του συστήματος είναι οντοκεντρικό από την φύση του αλλά και από τον σχεδιασμό του. Aπό την στιγμή που κάθε προγραμματιστής αναλαμβάνει ορισμένα modules κάθε φορά, η διαχείριση και η αποσφαλμάτωση του κώδικα δεν αποτελεί πρόβλημα. Για τους ίδιους λόγους, το σύστημα είναι επεκτάσιμο, καλύπτοντας κάθε ιδιαίτερη περίπτωση χρήσης του πελάτη, χωρίς ιδιαίτερο πρόβλημα αναπροσαρμογής του κώδικα.
\subsection{Επιχειρηματικά Ρίσκα}
Η ανάπτυξη του Safeguard γίνεται με την μεθοδολογία Agile, κρατώντας τον πελάτη ενημερωμένο για την πορεία του έργου και του προσφέρει την δυνατότητα επαναδιευκρίνισης των απαιτήσεών του έγκαιρα. Το σύστημα θα μπορεί να προσαρμοστεί στις ανάγκες μελλοντικών πελατών όπως περιγράφεται παραπάνω. \\
	To Safeguard είναι σχεδιασμένο για να προσελκύει, χάρη στο user friendly περιβάλλον του, την απλότητα και ταυτόχρονα την αποτελεσματικότητά του. Μαζί με την εγκατάσταση θα γίνεται και παρουσίαση της χρήσης του συστήματος, κάνοντάς το ακόμα πιο προσιτό στον ενδιαφερόμενο. Εφόσον αυτά επιτευχθούν, το πελατολόγιο αλλά και οι επιχειρηματικές ευκαιρίες, θα διευρυνθούν σημαντικά.
\subsection{Αναπτυξιακά Ρίσκα}
Η ανάπτυξη του λογισμικού βασίζεται σε ευρέως χρησιμοποιούμενες τεχνολογίες και εργαλεία. Ό,τι βιβλιοθήκες και συμπληρωματικό πρόγραμμα χρησιμοποιούμε, έχουμε φροντίσει να είναι stable λογισμικό και να παρέχεται αρκετό documentation και παραδείγματα, για να μειώσουμε τον χρόνο ανάπτυξης και τα λάθη που μπορούν να αποφευχθούν.
\section{Legal Feasibility (Νομική εφικτότητα)}
Καθώς τα εργαλεία και το λογισμικό που χρησιμοποιούνται είναι δωρεάν και open-source, δεν θα υπάρξει πρόβλημα στο licensing του τελικού προϊόντος. \\
Το προϊόν και η εγκατάστασή του, προορίζεται για ιδιωτικούς χώρους και τα καταγραφέντα δεδομένα, εφόσον το επιθυμεί ο πελάτης, παραμένουν σε υπολογιστή του, που δεν έχουμε πρόσβαση. Έτσι εξασφαλίζουμε πως δεν θα υπάρχει κάποια νομική επίπτωση περί προσωπικών δεδομένων και ιδιωτικότητας στην χρήση του προϊόντος / υπηρεσίας που προσφέρουμε, και με την υπηρεσία της τοπικής εγκατάστασης server, ο χρήστης θα είναι σίγουρος πως τα δεδομένα του δεν καταγράφονται ή χρησιμοποιούνται από την εταιρεία. \\
Αφού δεν παραβιάζεται η ιδιωτικότητα των χρηστών και το Safeguard χτίζεται σε δωρεάν τεχνολογίες, θεωρείται πως είναι νομικά εφικτό (Legally Feasible).
\section{Εργαλεία}
Χρησιμοποιήθηκαν:
\begin{itemize}
    \item \LaTeX/Overleaf.com - Συγγραφή του παρόντος τεχνικού κειμένου
    \item Photoshop - Φωτογραφία Σελίδας Τίτλου
    \item Google Docs - Συνεργατική συγγραφή του κειμένου
\end{itemize}
\end{document}