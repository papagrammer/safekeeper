\documentclass{article}
\usepackage[LGR,T1]{fontenc}
\usepackage[utf8]{inputenc}
\usepackage{lmodern,textcomp}
\usepackage[greek, english]{babel}
\usepackage{alphabeta}
\usepackage{natbib}
\usepackage{graphicx}
\usepackage{biblatex}
\usepackage{tabularx}
\usepackage{longtable}
\addbibresource{references.bib}
\renewcommand\tabularxcolumn[1]{m{#1}}% for vertical centering text in X column

\def\code#1{\texttt{#1}}

\usepackage{eso-pic}% http://ctan.org/pkg/eso-pic
\usepackage{lipsum}% http://ctan.org/pkg/lipsum

\usepackage[section]{placeins}

\title{Risk Assessment - v0.1}

\author{\\
\includegraphics[width=3in]{safeguard}\\[1ex]\\\\
}

\begin{document}

\maketitle
\newpage

\begin{center}
Λασκαρέλιας Βάιος - laskarelias@ceid.upatras.gr - ΑΜ 1054432 : Editor \\
Πάπα Αντόν - papa@ceid.upatras.gr - ΑΜ 1054337 : Editor
\end{center}

\begin{center}
\begin{tabular}{|l|c|c|}
\hline
Ονοματεπώνυμο & email & Αριθμός μητρώου \\
\hline
Θεόδωρος Ντάκουρης & ntakouris@ceid.upatras.gr & 1054332 \\
Βασίλειος Βασιλόπουλος & vvasil@ceid.upatras.gr &  1054410\\
Νικόλαος Σουλτάνης & soultanis@ceid.upatras.gr & 1054319  \\
Βάιος Λασκαρέλιας & laskarelias@ceid.upatras.gr & 1054432 \\
Αντόν Πάπα & papa@ceid.upatras.gr & 1054337 \\
\hline
\end{tabular}
\end{center}

\renewcommand{\contentsname}{Περιεχόμενα}

\tableofcontents
\newpage

\thispagestyle{plain} % empty
\mbox{}
\newpage

\section{Πιθανά ρίσκα και επιλύσεις κατά την ανάπτυξη}
Παρουσιάζεται παρακάτω ο πίνακας των πιθανών ρίσκων που θα κληθούμε να αντιμετωπίσουμε στην διάρκεια της ανάπτυξης της πρώτης έκδοσης και επιπλέον:
\begin{itemize}
\item η πιθανότητα να συμβούν.
\item η σοβαρότητά τους για την βιωσιμότητα του project.
\item η πιθανή καθυστέρηση που προκαλούν στο λανσάρισμα του προϊόντος ή ακόμα και η τελμάτωση του project. 
\item κάποιες πιθανές μέθοδοι πρόληψης και αντιμετώπισης αυτών.
\end{itemize}

\begin{table}[!htbp]
\makebox[\linewidth]{
\begin{tabularx}{0.9\paperwidth}{ ||X|s|s|s|X|| } 
 \hline
 Ρίσκο & Πιθανότητα & Σοβαρότητα & Ζημία & Πρόληψη και αντιμετώπιση  \\ 
 \hline
 \hline

 Έλλειψη εξοπλισμού & Μικρή & Μικρή & 3 - 7 ημέρες & Xρήση δοκιμασμένων εμπορικών τεχνολογιών για την αποφυγή άσκοπου testing \\
 \hline
 Βλάβη εξοπλισμού απαραίτητου για τηλεεργασία & Μικρή & Μεσαία & 3 - 7 ημέρες & Υπάρχων εφεδρικός εξοπλισμός όπου δύναται και άμεση αποκατάσταση της βλάβης ή αντικατάσταση του εξοπλισμού \\
 \hline 
  Σοβαρή ασθένεια μέλους & Μεσαία & Μεσαία & 7 - 14 ημέρες & Αναδιοργάνωση φόρτου εργασίας, υποστήριξη του ασθενούς (αναρρωτική άδεια με αποδοχές) και καθημερινή ενημέρωσή του για την πρόοδο, εφόσον δύναται \\
 \hline
 Υποτίμηση χρονικών περιθωρίων και φόρτου εργασίας & Μεγάλη & Μεσαία & 3 - 30 ημέρες & Τακτικά meetings για κάλυψη κενών, ετοιμότητα των μελών για ανάληψη περαιτέρω φόρτου εργασίας, ενδεχομένως διαφορετικής φύσεως ανά πάσα στιγμή, εργασία από το σπίτι \\
 \hline 
 Αδυναμία εύρεσης επιπλέον πελατών & Μικρή & Μεγάλη & Καταστροφική & Αλλαγή προσέγγισης της προώθησης του project και του στοχευμένου κοινού, έρευνα αγοράς για την προσθήκη δημοφιλών δυνατοτήτων \\ 
 \hline
 Νομικά προβλήματα ιδιωτικότητας & Μικρή & Μεγάλη & Καταστροφική & Κατάλληλο licensing του τελικού προϊόντος και δυνατότητα χρήσης ιδιωτικού server του πελάτη. \\ 
 \hline
 \end{tabularx}
}
\end{table}

\newpage

\section{Πιθανά ρίσκα και επιλύσεις μετά την πρώτη έκδοση}

Σε αυτήν την περίπτωση, το ρίσκο προκύπτει μετά το λανσάρισμα της πρώτης έκδοσης του προγράμματος, και προκαλεί μόνο χρηματική ζημία για την εταιρία που δεν μπορούμε να προβλέψουμε.

\begin{table}[!htbp]
\makebox[\linewidth]{
\begin{tabularx}{0.9\paperwidth}{ ||X|s|s|X|| } 
 \hline
 Ρίσκο & Πιθανότητα & Σοβαρότητα & Πρόληψη και αντιμετώπιση  \\ 
 \hline
 \hline
 Απρόβλεπτο ανταγωνιστικό προϊόν & Μεσαία & Μεσαία & Έγκαιρο λανσάρισμα στην αγορά, ανταγωνιστική τιμή και user experience σε σχέση με άλλες λύσεις. Ιδιαίτερη σημασία στην σταθερότητα και την ασφάλεια του τελικού προϊόντος \\
 \hline
 Αδυναμία άμεσης εγκατάστασης σε νέους πελάτες λόγω ζήτησης & Μικρή & Μεσαία &  Προτεραιότητα στην ταχύτητα εξυπηρέτησης των πρώτων πελατών αντί για το κόστος του εξοπλισμού για την αποφυγή αρνητικής διαφήμισης, έγκαιρες προπαραγγελίες και δημιουργία συνεργασιών με προμηθευτές \\
 \hline
 Αδυναμία ανταπόκρισης του server λόγω απρόσμενου φόρτου & Μεσαία & Μεσαία & Έγκαιρη εξασφάλιση resource allocation από την μεριά του hosting service και προοδευτική αναβάθμιση των υπηρεσιών \\
 \hline
 Αδυναμία υποστήριξης τελικού προϊόντος λόγω πολλών διαφορετικών setups & Μικρή & Μεγάλη & Σωστή παραμετροποίηση του κώδικα και επικοινωνία με τους πελάτες με custom setup εφόσον είναι εφικτό. \\
 \hline
 Επικοινωνιακά προβλήματα με τους πελάτες & Μεσαία & Μεγάλη &  ανάθεση κατάλληλου μέλους για διαχείριση επικοινωνίας και πρόσληψη γραμματέα εάν χρειαστεί \\
 \hline

 \end{tabularx}
}
\end{table}

\newpage

\section{Χρηματική ζημία και απρόβλεπτα προβλήματα}
Υπολογίζεται πως κάθε μέρα εργασίας ανά άτομο κοστίζει περίπου \texteuro50. \\ \\
Οι καθυστερήσεις στα παραδοτέα προς τον πελάτη, ιδιαιτέρως στο τελικό στάδιο, μπορεί να αποβούν και καταστροφικές για την πορεία του project, ή ακόμα και να το καταστήσουν μη βιώσιμο. Σε περιπτώσεις σοβαρής έλλειψης εξοπλισμού ή μεγάλης ζήτησης θα πρέπει να δημιουργήσουμε συνεργασίες με προμηθευτές, αποθήκες και προσωπικό εγκατάστασης και παρουσίασης του συστήματος στους πελάτες, πράγμα που προσθέτει στα λειτουργικά κόστη της επιχείρησης και θα είναι δύσκολο να καλυφθούν αρχικά. Κατά την ανάπτυξη του συστήματος θα δοθεί ιδιαίτερη σημασία στην ασφάλεια και την ευκολία χρήσης του, για την αποφυγή σημαντικής απώλειας πελατών λόγω ανταγωνιστικών προϊόντων και υπάρχοντων λύσεων. Σημαντική είναι επίσης η επεκτασιμότητα του συστήματος πέραν των αρχικών του λειτουργιών, έτσι ώστε να προσεγγιστούν αργότερα περισσότεροι χρήστες και ταυτόχρονα να καλύπτουμε τις ανάγκες που δημιουργούν οι υπάρχοντες πελάτες.

\section{Εργαλεία}
Χρησιμοποιήθηκαν:
\begin{itemize}
    \item \LaTeX/Overleaf.com - Συγγραφή του παρόντος τεχνικού κειμένου
    \item Photoshop - Φωτογραφία Σελίδας Τίτλου
    
\end{itemize}
\end{document}